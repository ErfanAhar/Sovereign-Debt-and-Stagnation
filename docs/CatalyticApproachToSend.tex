%2multibyte Version: 5.50.0.2960 CodePage: 1252

\documentclass{article}
%%%%%%%%%%%%%%%%%%%%%%%%%%%%%%%%%%%%%%%%%%%%%%%%%%%%%%%%%%%%%%%%%%%%%%%%%%%%%%%%%%%%%%%%%%%%%%%%%%%%%%%%%%%%%%%%%%%%%%%%%%%%%%%%%%%%%%%%%%%%%%%%%%%%%%%%%%%%%%%%%%%%%%%%%%%%%%%%%%%%%%%%%%%%%%%%%%%%%%%%%%%%%%%%%%%%%%%%%%%%%%%%%%%%%%%%%%%%%%%%%%%%%%%%%%%%
\usepackage{amsfonts}
\usepackage{amsmath}

\setcounter{MaxMatrixCols}{10}
%TCIDATA{OutputFilter=LATEX.DLL}
%TCIDATA{Version=5.50.0.2960}
%TCIDATA{Codepage=1252}
%TCIDATA{<META NAME="SaveForMode" CONTENT="1">}
%TCIDATA{BibliographyScheme=Manual}
%TCIDATA{Created=Wednesday, November 15, 2023 18:40:15}
%TCIDATA{LastRevised=Tuesday, November 21, 2023 17:59:42}
%TCIDATA{<META NAME="GraphicsSave" CONTENT="32">}
%TCIDATA{<META NAME="DocumentShell" CONTENT="Standard LaTeX\Blank - Standard LaTeX Article">}
%TCIDATA{CSTFile=40 LaTeX article.cst}

\newtheorem{theorem}{Theorem}
\newtheorem{acknowledgement}[theorem]{Acknowledgement}
\newtheorem{algorithm}[theorem]{Algorithm}
\newtheorem{axiom}[theorem]{Axiom}
\newtheorem{case}[theorem]{Case}
\newtheorem{claim}[theorem]{Claim}
\newtheorem{conclusion}[theorem]{Conclusion}
\newtheorem{condition}[theorem]{Condition}
\newtheorem{conjecture}[theorem]{Conjecture}
\newtheorem{corollary}[theorem]{Corollary}
\newtheorem{criterion}[theorem]{Criterion}
\newtheorem{definition}[theorem]{Definition}
\newtheorem{example}[theorem]{Example}
\newtheorem{exercise}[theorem]{Exercise}
\newtheorem{lemma}[theorem]{Lemma}
\newtheorem{notation}[theorem]{Notation}
\newtheorem{problem}[theorem]{Problem}
\newtheorem{proposition}[theorem]{Proposition}
\newtheorem{remark}[theorem]{Remark}
\newtheorem{solution}[theorem]{Solution}
\newtheorem{summary}[theorem]{Summary}
\newenvironment{proof}[1][Proof]{\noindent\textbf{#1.} }{\ \rule{0.5em}{0.5em}}
\input{tcilatex}
\begin{document}


\section{\textbf{A two period model:}}

The economy is populated by a representative agent that draws utility from
consumption in each period, and by a continuum of risk-neutral foreign
lenders.

The representative agent preferences are given by 
\begin{equation*}
u(c_{1})+\beta \mathbb{E}u(c_{2}),
\end{equation*}
where $u$ is strictly increasing, strictly concave and satisfies standard
Inada conditions.

The initial wealth of the agent is denoted by $\varepsilon ,$ which we take
it to be arbitrarily small. The endowment in the second period is
distributed according to 
\begin{equation*}
y_{2}=%
\begin{cases}
y^{l}, & \text{ with probability }p \\ 
y^{h}, & \text{ with probability }\left( 1-p\right)%
\end{cases}%
\end{equation*}%
in which $y^{l}<y^{h}$. As the initial wealth is low, the agent will want to
borrow.

\subsection{The "catalytic" lender of last resort.}

We assume there is an international non-for profit institution (the European
Central Bank or the International Monetary Fund) that is willing to lend an
amount $b^{o}$ at an interest rate $R^{o}.$ We will focus only on cases in
which $R^{o}\geq R^{\ast },$ where $R^{\ast }$ is an international risk free
rate.

We will study how the set of equilibria depends on these two policy
instruments, $(b^{o},R^{o}).$ We denote this international institution as
LOLR

\subsection{Private Lenders}

Assume first that $R^{o}=R^{\ast }.$

In period one, the borrower moves first and issues a non-contingent debt
level $b$. Lenders respond with an interest rate $R$. As $R=R^{\ast },$
without loss of generality, the country would first borrow from the LOLR up
to the maximum, and then will borrow the rest from private agents.

We denote by $R(b,b^{o})$ the interest rate schedule faced by the borrower.
In period two, after observing the endowment $y_{2}$, the borrower decides
whether to pay the debt or to default. In case of repayment, the borrower
consumes the endowment net of debt repayment, $c_{2}=y_{2}-\left[ R^{\ast
}b^{o}+R(b-b^{o})\right] $. In case of default, consumption is $c_{2}=y^{d}$%
. The agent defaults if the cost of repayment is larger than the benefit: 
\begin{equation*}
\underbrace{\left[ R^{\ast }b^{o}+R(b-b^{o})\right] }_{\text{cost of
repayment}}>\underbrace{y_{2}-y^{d}}_{\text{benefit of repayment}}.
\end{equation*}

In the first period, given initial wealth $\varepsilon $ and an interest
rate schedule $R(b,b^{o})$, the borrower solves the following problem 
\begin{align}
V(\omega )& =\max_{b}\left\{ u(c_{1})+\beta \mathbb{E}u(c_{2})\right\} ,
\label{eq:HH} \\
\text{subject to~~}& c_{1}=\omega +b,  \notag \\
& c_{2}=\max \left\{ y_{2}-\left[ R^{\ast }b^{o}+R(b-b^{o})\right]
,y^{d}\right\} ,  \notag
\end{align}%
and subject to a maximum debt level constraint, $b\leq \overline{B}$.

The assumption that the borrower moves first by choosing a level of debt and
that lenders move next with an interest rate schedule is standard. We depart
from the literature, as in \citet{Aguiar+Gopinath:2006} and %
\citet{Arellano:2008}, in that we assume that the borrower chooses current
debt $b$, rather than debt at maturity, $Rb$. The risk-neutral lenders will
be willing to lend to the agent as long as the expected return is the same
as the risk-free rate $R^{\ast }$, that is, 
\begin{equation}
R^{\ast }=h(R;b,b^{o})\equiv \left[ 1-\Pr \left( y_{2}-y^{d}<\left[ R^{\ast
}b^{o}+R(b-b^{o})\right] \right) \right] R,  \label{eq:exp_ret1}
\end{equation}%
in which $h(R;b,b^{o})$ is the expected return to the lender when the
interest rate is $R$. Given a value for $b$, the expected return for lenders
can be written as 
\begin{equation}
h(R;b)=%
\begin{cases}
R, & \text{ if }\left[ R^{\ast }b^{o}+R(b-b^{o})\right] \leq y^{l}-y^{d} \\ 
R\text{ }(1-p), & \text{ if }y^{l}-y^{d}<\left[ R^{\ast }b^{o}+R(b-b^{o})%
\right] \leq y^{h}-y^{d} \\ 
0, & \text{ if }\left[ R^{\ast }b^{o}+R(b-b^{o})\right] >y^{h}-y^{d}.%
\end{cases}
\label{eq:exp_ret2}
\end{equation}

\subsubsection{The case $b^{o}=0.\,$}

In Figure \ref{fig:exp_return}, we plot the expected return as a function of
the interest rate $R$, for three levels of debt, together with the risk-free
rate $R^{\ast }$. Notice that for low levels of $R$, the expected return is
equal to $R$ since debt is repaid with probability one. In this region, as $%
R $ increases, the expected return increases one to one. Eventually, $R$
will be high enough that the borrower will default in the low output state,
which happens with probability $p$. At this point, the expected return jumps
down. As $R$ increases, the expected return increases at a lower rate, $%
(1-p) $, since repayment happens only in the high output state. Finally, for
high enough $R$, default will happen with probability one and the expected
return will be zero. A higher level of debt decreases the expected return,
uniformly, shifting the curves downwards.

%% Figure - expected return (different b's)
%\begin{figure}[h!]
%\caption{Expected return function for different levels of debt}\label{fig:exp_return}
%\subfloat[low debt]{ \includegraphics[scale=1.0]{Figures_2States2Periods/Figures/fig_exp_return_low_b} \label{fig:exp_return_low_b}} \\[1.5ex]
%\subfloat[medium debt]{ \includegraphics[scale=1.0]{Figures_2States2Periods/Figures/fig_exp_return_medium_b} \label{fig:exp_return_medium_b}} \\[1.5ex]
%\subfloat[high debt]{ \includegraphics[scale=1.0]{Figures_2States2Periods/Figures/fig_exp_return_high_b} \label{fig:exp_return_high_b}}
%\end{figure}

For low levels of debt, there is only one solution to equation %
\eqref{eq:exp_ret1}, with $R=R^{\ast}$. For intermediate levels of debt,
there are two solutions: one solution has $R=R^{\ast}$ associated with zero
probability of default, the other has $R=R^{\ast}/(1-p)$ associated with
probability of default equal to $p$. For higher levels of debt, the only
solution is the high rate $R=R^{\ast}/(1-p)$. Finally, for even higher debt,
there is no solution. There are multiple solutions only for intermediate
levels of debt.

We can now define the following correspondence relating debt levels to
interest rates, 
\begin{equation}  \label{eq:correspondence}
\mathcal{R}(b) = 
\begin{cases}
R^{\ast }, & \text{ if } b\leq \frac{y^{l}-y^{d}}{R^{\ast }} \\ 
\frac{R^{\ast }}{1-p}, & \text{ if } \frac{y^{l}-y^{d}}{R^{\ast }/(1-p)}%
<b\leq \frac{y^{h}-y^{d}}{R^{\ast }/(1-p)} \\ 
\infty , & \text{ if }b>\frac{y^{h}-y^{d}}{R^{\ast }/(1-p)}%
\end{cases}%
\end{equation}

An equilibrium is an interest rate schedule $R(b)$ and a debt policy
function $b(\omega)$ such that, given the schedule, the debt policy function
solves the problem of the borrower in \eqref{eq:HH}, and the schedule $R(b)$
is a selection of the correspondence $\mathcal{R}(b)$.

The correspondence $\mathcal{R}(b)$ is plotted in Figure \ref{fig:schedules}%
. For all debt levels below $b_{1} \equiv \frac{y^{l}-y^{d}}{R^{\ast }/(1-p)}
$, there is only one interest rate, the risk-free rate. For debt levels
between $b_{1}$ and $b_{2} \equiv \frac{y^{l}-y^{d}}{R^{\ast }}$, there are
two possible interest rates, the risk-free rate and a high rate. For debt
levels between $b_{2}$ and $\overline{b}\equiv \frac{y^{h}-y^{d}}{R^{\ast
}/(1-p)}$ there is again only one interest rate, the high rate. There are
multiple interest rate schedules that can be selected from this
correspondence. We focus on two of those schedules: a low interest rate
schedule, $R^{low}(b)$ in Figure \ref{fig:schedule_low}, and a high interest
rate schedule, $R^{high}(b)$ in Figure \ref{fig:schedule_high}. We think of $%
b_{1}$ as the debt level above which interest rates jump because of
expectations, since alternative expectations could sustain low interest
rates. We think of $b_{2}$ as the debt level above which interest rates jump
because of fundamentals, since no expectations could sustain lower interest
rates.

Whether spreads are low or high has implications for the level of debt that
can be raised. The region of multiplicity happens for intermediate levels of
debt, between $b_{1}$ and $b_{2}$. If debt is sufficiently low, interest
rates can only be low, while if debt is sufficiently high, rates can only be
high. It is for intermediate levels of debt that interest rates can be
either high or low depending on expectations.

%% Figure - interest rate schedules
%\begin{figure}[h!]
%\caption{Interest rate schedules}\label{fig:schedules}
%\subfloat [low interest rate schedule \label{fig:schedule_low}]{ \includegraphics[scale=1.0]{Figures_2States2Periods/Figures/fig_schedule_low}} \\[0ex]
%\subfloat [high interest rate schedule \label{fig:schedule_high}]{ \includegraphics[scale=1.0]{Figures_2States2Periods/Figures/fig_schedule_high}}
%\end{figure}
%
%% Figure - policy function
%\begin{figure}[h!]
%\caption{Debt policy function}\label{fig:polfcn}
%\includegraphics[scale=1.0]{Figures_2States2Periods/Figures/fig_polfcn2}
%\end{figure}

Figure \ref{fig:polfcn} shows the optimal debt policy as a function of the
initial wealth for the high and low interest rate schedules. For high levels
of wealth, the optimal choice of debt is below $b_{1}$ regardless of which
schedule the borrower is facing. As wealth decreases, the schedule matters.
For the high interest rate schedule, the borrower chooses to keep debt
levels at $b_{1}$ in order to avoid the discrete jump in interest rates on
the whole level of debt. Eventually, for low enough wealth, the marginal
utility in the first period is high enough that the borrower chooses to
increase its debt level discretely. This discrete jump shows that the
borrower has incentives to avoid at least part of the multiplicity region
between $b_{1}$ and $b_{2}$. As wealth decreases even more, debt levels keep
increasing until they reach the borrowing limit $\overline{b}$. When facing
the low interest rate schedule, borrowing keeps on increasing as wealth
declines until it reaches the level $b_{2}$. At this point, there is a
choice to keep it constant for lower levels of wealth. Eventually, there is
also a discrete jump, and debt continues to increase until they reach the
borrowing limit.

The choice of keeping debt levels constant as wealth decreases is a form of
endogenous austerity. This happens in our model because of the discrete
jumps in interest rates induced by both expectations and

\subsubsection{The case $b^{o}>0.\,$}

As we are interested in exploring policies that may rule out bad equilibria
without the LOLR\ loosing money, we will focus only on the cases for which
there is multiple equilibria, as depicted in Figure XXX. Notice that the
good equilibrium implies that there is never default, so the equilibrium
interest rate faced by the country satisfies $R=R^{\ast }.$

Consider now the case in which the policy parameter $b^{o}$ is strictly
positive. As before, we first study the function $h(R;b,b^{o}),$ which is
the expected return to the lender when the interest rate is $R.$

As shown in Figure XXX, there are two thresholds that matter: the first is
the value for $R$ such that the country is indifferent between paying the
debt and defaulting in the low income state. For lower values of this
threshold, the country pays in both states of nature and the return is $R.$
The second threshold is the - higher - value for $R$ such that the country
is indifferent between \bigskip paying the debt and defaulting in the high
income state. The country only defaults in the low income state if $R$ is in
between the thresholds, so the expected return is $(1-p)R.$ Finally, the
country defaults in every state for interest rates higher that the second
threshold, so the return is zero.

To understand the effect of the policy parameter $b^{o},$ we therefore only
need to understand how it changes the two thresholds.

The first threshold is defined by the value for $R^{T_{1}}$ equality 
\begin{equation*}
R^{\ast }b^{o}+R^{T_{1}}(b-b^{o})=y^{l}-y^{d},
\end{equation*}%
which implies 
\begin{equation*}
R^{T_{1}}=\frac{y^{l}-y^{d}-R^{\ast }b^{o}}{\left( b-b^{o}\right) }%
=R^{T_{1}}(b,b^{o}).
\end{equation*}%
Notice that%
\begin{equation*}
\frac{\partial R^{T_{1}}}{\partial b^{0}}=\frac{\left[ y^{l}-y^{d}-bR^{\ast }%
\right] }{\left( b-b^{o}\right) ^{2}}
\end{equation*}%
As we assumed that $b$ was such that the good - and the bad - equilibrium
exists, then $\left[ y^{l}-y^{d}-bR^{\ast }\right] >0,$ so the derivative is
positive. Thus, as $b^{0}$ is increased, the threshold $R^{T_{1}}$ moves to
the rigth - i.e., it increases.

Thus, this "catalytic" policy has the same effect on the thresholds as a
reduction in total debt. Thus, it moves the multiplicity region to the
right. Therefore, there is a potential effect on the equilibrium. Imagine,
for instance that the the original equilibrium had the country borrowing in
the multiplicity region, at high spreads, when $b^{0}=0.$ Using the language
adopted above, in this equilibrium the country was "gambling for
redemption", at high rates. Imagine, on the other hand, that the country was
almost indiferent betwen this action and "endogenous austerity". Then, it
follows that a relatively low value of "catalytic" lending can push the
country towards "endogenous austerity" and low rates.

\bigskip

Conclusions:

1. The "whatever it takes"\ policy rules out multiple equilibria.

2. The "catalytic" approach moves the multiplicity region and therefore it
can, well designed, affect the country's actions and therefore the
equilibrium interest rates. It is conceivable that, as the new rates are
lower, the country ends up borrowing even more that in the bad equilibrium
with $b^{0}=0.$

\bigskip

A final point:\ Imagine now that - as it is in reality - an agreement with
the LOLR implies a commitment to a maximum amount the country can borrow,
but it is the choice of the country to tap on those finds or not.\footnote{%
Incidentally, during 1997 and 1998, Argentina had an aggrement with the
Fund, but did not tap on those, it chose to issue bonds in private markets.
I\ also think - this must be checked - that Clinton's and the IMF\ support
package for Mexico, that totalled about 30 billion I\ believe, was not
totally tapped - and it weas returned completely in about a year.
\par
{}}

So far, I\ have assumed that $R^{o}=R^{\ast }.$ Imagine now that $%
R^{o}=R^{\ast }+\delta ,$ for very small $\delta .$ Clearly, as the
equilibrium is either $R=R^{\ast }$ or $R=\frac{R^{\ast }}{1-p},$ in the
good equilibrium, the country would never tap on the funds of the LOLR,
since it charges a higher rate. However, I\ conjecture that the commitment
to lend at $R^{o}=R^{\ast }+\delta $ moves the return function $h(R;b,b^{o})$
in a similar fashion, so it would still imply moving the multiplicity region
to the right, indusing endogenous austerity, even though the funds are never
used. One way to prove this conjecture would be by taking the limit when $%
\delta \rightarrow 0.$

\bigskip

\end{document}
